\section{再短一点}
\begin{frame}[fragile]
\frametitle{数据类}
\begin{quotebox}[penrosered]
    构造一个类来传输数据
\end{quotebox}
\begin{javacode}[basicstyle=\scriptsize\ttfamily,emph={[1]Customer}]
    // 好多行
    public class Customer {
        private String mName;
        private String mEmail;
        public Customer(String name, String email) {
            this.mName = name;
            this.mEmail = email;
        }
        public String getName() { return mName; }
        public void setName(String name) { this.mName = name }
        public String getEmail() { return mEmail; }
        public void setEmail(String email) { this.mEmail = email }
    }
\end{javacode}
\end{frame}

\begin{frame}[fragile]
\frametitle{数据类}
\begin{quotebox}
    |data class| 来实现 Data transfer object
\end{quotebox}
\begin{kotlincode}[emph={[1]Customer}]
    // 一行
    data class Customer(val name: String, val email: String)
\end{kotlincode}
\end{frame}

\begin{frame}[fragile]
\frametitle{单例模式}
\begin{quotebox}[penrosered]
    饿汉式,非 lazy loading,线程安全
\end{quotebox}
\begin{javacode}[basicstyle=\scriptsize\ttfamily,emph={[1]MySingleton}]
    public final class MySingleton {
        private static final int myProperty = 3;
        private static final MySingleton INSTANCE = new MySingleton();
        private MySingleton() {}
        public final int getMyProperty() {
            return myProperty;
        }
        public final void myFunction() {
            return "Hello";
        }
    }
\end{javacode}
\end{frame}

\begin{frame}[fragile]
\frametitle{单例模式}
\begin{quotebox}
    |object| 来实现 Data transfer object
\end{quotebox}
\begin{kotlincode}[emph={[1]MySingleton}]
    object MySingleton {
        val myProperty = 3
        fun myFunction() = "Hello"
    }
\end{kotlincode}
\end{frame}

\begin{frame}[fragile]
\frametitle{集合容器}
\begin{quotebox}
    快速创建 |Array, List, Set, Map|
\end{quotebox}
\begin{kotlincode}
    val strings = arrayOf("a", "b", "c")
    // 不可变集合
    var numList = listOf(1, 2, 3)
    // 可变集合
    var numMap = mutableMapOf(1 to "one", 2 to "two", 3 to "three")
\end{kotlincode}
\nonumberfootnote{\faInfoCircle{} \texttt{to} 为生成 \texttt{Pair} 对象的中缀运算符}
\end{frame}

\begin{frame}[fragile]
\frametitle{函数式}
\begin{quotebox}[penrosegreen]
    下面的代码会输出什么?
\end{quotebox}
\begin{kotlincode}[emph={[1]sortedBy,filter,map,forEach}]
    val fruits = listOf("banana", "avocado", "apple", "kiwifruit")
    fruits
      .filter { it.startsWith("a") }
      .sortedBy { it }
      .map { it.toUpperCase() }
      .forEach { println(it) }
\end{kotlincode}
\nonumberfootnote{\faInfoCircle{} 像使用 \faPython{}\texttt{Pandas} 一样,同样有 \texttt{flatMap, groupBy, reduce} 等操作}
\end{frame}

\begin{frame}[fragile]
\frametitle{扩展函数和 \texttt{when} 表达式}
\begin{quotebox}
    扩展函数可以增加已有类的功能
\end{quotebox}
\begin{kotlincode}
    fun String.reply(): String =
        when (this) {
            "Hello"    -> "World"
            "Bye"      -> "Bye"
            else       -> "Unknown"
        }
    // 使用
    "Hello".reply() // "World"
\end{kotlincode}
\end{frame}

\begin{frame}[fragile]
\frametitle{作用域函数}
\begin{quotebox}
    用一种简洁明了的方式在对象的上下文中执行代码块
\end{quotebox}
\begin{kotlincode}[emph={[1]adam,result}]
    // 用 apply 来初始化
    val adam = Person("Adam").apply {
        it.age = 20  // it 可以省略
        it.city = "London"
    }
    // 初始化并返回计算结果
    val result = service.run {
        port = 8080
        query(prepareRequest() + " to port $port")
    }
\end{kotlincode}
\nonumberfootnote{\faInfoCircle{} \texttt{\$port} 字符串内插,甚至可以使用表达式}
\end{frame}

\begin{frame}[fragile]
\frametitle{作用域函数}
\begin{quotebox}
    不同的是这个对象在块中如何使用,以及整个表达式的结果是什么
\end{quotebox}

\begin{table}[]
\begin{tabular}{llll}
\textbf{函数} & \textbf{对象引用} & \textbf{返回值} & \textbf{是否是扩展函数}\\
\texttt{let}   & it   & Lambda 表达式结果 & 是\\
\texttt{run}   & this & Lambda 表达式结果 & 是\\
\texttt{run}   & -    & Lambda 表达式结果 & 不是:调用无需上下文对象\\
\texttt{with}  & this & Lambda 表达式结果 & 不是:把上下文对象当做参数\\
\texttt{apply} & this & 上下文对象        & 是\\
\texttt{also}  & it   & 上下文对象        & 是\\
\end{tabular}
\end{table}
\end{frame}


