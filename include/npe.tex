\section{和NPE说再见}
\begin{frame}[fragile]
    \frametitle{空引用异常}
    \begin{quotebox}[yzlred]
        Java 最常见的运行时异常
    \end{quotebox}
    \begin{javacode}[basicstyle=\scriptsize\ttfamily,emph={[1]sample}]
    String sample = null;
    System.out.println(sample.toString()); // NullPointerException!
    \end{javacode}
    \begin{quotebox}
        \kotlin{} 用类型系统消除代码中的 NPE
    \end{quotebox}
    \begin{kotlincode}[basicstyle=\scriptsize\ttfamily]
    var a: String = "abc"
    a = null // 编译错误
    var b: String? = "abd" // ? 表示变量可空
    val l = b.length // 编译错误,因为 b 可空
    val l = b?.length // 安全调用,l 的类型为可空整型 Int?
    // let 作用域函数,非空后执行
    l?.let { …… // 只有非空才会执行 }
    \end{kotlincode}
\end{frame}

\begin{frame}[fragile]
    \frametitle{Elvis 操作符}
    \begin{quotebox}
        \textcolor{yzlblue}{\texttt{?:}}用来缩写 |if not null and else|
    \end{quotebox}
    \begin{kotlincode}[basicstyle=\scriptsize\ttfamily]
    // 返回值
    val files = File("Test").listFiles()
    println(files?.size ?: "empty")
    // 执行语句
    val values = ……
    val email = values["email"] ?: throw IllegalStateException("Email is missing!")
    \end{kotlincode}
    \nonumberfootnote{\faInfoCircle{}\texttt{?:}如果侧过头,可以看到猫王}
\end{frame}

\begin{frame}[fragile]
    \frametitle{类型转换}
    \begin{kotlincode}[basicstyle=\scriptsize\ttfamily]
    // !! 非空断言操作符,强制转换,若空会抛出 NPE
    val l = b!!.length
    // 用 as? 安全转换
    val aInt: Int? = a as? Int
    // 智能转换可空为非空
    val p = Person(first = "Zhiling", middle = null, last = "Ye")
    if (p.middle != null) {
        val middleNameLength = p.middle.length // 这里变量的类型为 Int
    }
    // 集合过滤非空元素
    val nullableList: List<Int?> = listOf(1, 2, null, 4)
    val intList: List<Int> = nullableList.filterNotNull()
    \end{kotlincode}
    \nonumberfootnote{\faExclamationCircle{}如非必要,请勿使用\textcolor{yzlred}{\texttt{!!}}操作符}
\end{frame}

