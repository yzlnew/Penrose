\section{参考}
\begin{frame}[fragile]
\frametitle{参考资料}
\newcommand\BOOK[1]{\textbf{#1}}
\footnotesize
\begin{thebibliography}{99}
  \bibitem{}
  JetBrains.
    \newblock \BOOK{Kotlin 中文文档}
    \newblock 链接:\link{https://www.kotlincn.net/docs/reference/}
  \bibitem{}
  Ken Kousen
    \newblock \BOOK{Kotlin Cookbook: A Problem-Focused Approach}
    \newblock 购买:\href{https://item.jd.com/12695637.html}{\faShoppingCart}
  \bibitem{}
  Google.
    \newblock \BOOK{Andorid Developers}
    \newblock 链接:\link{https://developer.android.com/kotlin/}
\end{thebibliography}
\nonumberfootnote{\faInfoCircle 实际查阅的为乔禹昂的译作《Kotlin编程实践》}
\end{frame}

\begin{frame}[c]{关于}
\begin{table}[]
\begin{tabular}{cll}
\faPaintRoller & \textbf{主题}     & penrose            \\
\faFont        & \textbf{正文字体} & 更纱黑体 + Roboto  \\
\faTextWidth   & \textbf{等宽字体} & \texttt{JetBrains Mono} \\
\end{tabular}
\end{table}
\vspace{0.4cm}
\begin{center}
    \small{Creative Commons Attribution-ShareAlike 4.0 International}\\
  \huge
  \faCreativeCommons\,\faCreativeCommonsBy\,\faCreativeCommonsSa
\end{center}
\end{frame}

\begin{frame}[c,fragile]
\frametitle{调色板}
\begin{center}
\colorpalette[primary palette={penroseblue,penrosegreen,penrosegray}, secondary palette={},
primary variation=logoyellow, init text color=white, title text color=logoblack, title font=\bfseries, variation font=\ttfamily]{}
\colorpalette[primary palette={penrosered,penroseorange,penrosepink}, secondary palette={},
primary variation=logoblack, init text color=white, title text color=logoyellow, title font=\bfseries, variation font=\ttfamily]{}
\end{center}
\end{frame}

\begin{frame}[standout]
\begin{center}
\end{center}
\begin{center}
    \begin{tikzpicture}
        \node[anchor=center,inner sep=0,opacity=0.5] at (current page.center) {\includegraphics[width=4cm]{res/penrose_color.pdf}};
        \node[align=center,logoblack,font={\Huge\ttfamily}] at (current page.center) {Happy \textit{\textbackslash Coding}};
    \end{tikzpicture}
\end{center}
\end{frame}
